% !TEX root = main.tex

\newcommand{\Sec}[1]{Sec.~\ref{#1}}
\newcommand{\Thm}[1]{Thm.~\ref{#1}}
\newcommand{\Obs}[1]{Obs.~\ref{#1}}
\newcommand{\Prop}[1]{Prop.~\ref{#1}}
\newcommand{\Cor}[1]{Cor.~\ref{#1}}
\newcommand{\Lem}[1]{Lem.~\ref{#1}}

\paragraph{Isomorphism problems in light of Babai's breakthrough on Graph 
Isomorphism.}
In late 2015, Babai presented a quasipolynomial-time algorithm for  
\GIlong (\GI) \cite{Bab16}. This is widely regarded as one of the major 
breakthroughs in 
theoretical computer science of the past decade. Indeed, \GI
 has been at the heart of complexity theory 
nearly since its inception: both Cook and Levin were thinking about \GI when they 
defined $\cc{NP}$ \cite[Sec. 1]{AllenderDas}, \algprobm{Graph (Non-)Isomorphism} 
played a special role in the creation of the 
class $\cc{AM}$ \cite{babai85, GMR85, BM88}, 
and it still stands today 
as one of the few 
natural candidates for a problem that 
is ``$\cc{NP}$-intermediate,'' that is, in $\cc{NP}$, but neither in $\cc{P}$ nor 
$\cc{NP}$-complete \cite{Ladner} (see \cite{StackExchangeIntermediate} for 
additional 
candidates). Beyond its practical applications (e.\,g., \cite{SV17, irniger} and references therein) and its 
naturality, part of its fascination comes from its universal property: \GI is 
universal for isomorphism problems for ``explicitly given'' structures \cite[Sec.~15]{ZKT}, that is, first-order structures on a set $V$ 
where, e.\,g., a $k$-ary relation on $V$ is given by listing out a subset $R 
\subseteq V^k$. 

In light of Babai's breakthrough
on \GI \cite{Bab16}, it is natural to consider ``what's next?'' 
for isomorphism problems. That is, what isomorphism problems stand as crucial 
bottlenecks to further improvements on \GI, and what isomorphism problems should 
naturally draw our attention for further exploration? Of course, 
one of the main open questions in the area remains whether or not \GI is in $\cc{P}$.
Babai \cite[arXiv version, Sec.~13.2 and 13.4]{Bab16} already lists several 
isomorphism problems for further study, including \GpIlong, 
\algprobm{Linear Code Equivalence}, and \algprobm{Permutation Group Conjugacy}. In 
this paper we expand this list in what we argue is a very natural direction, 
namely to \emph{isomorphism problems for multi-way arrays}, also known as 
tensors.\footnote{There have been some disputes on the terminologies; see the 
preface of \cite{Lan12}. Our approach is to use 
multi-way arrays as the basic 
underlying object, and to use tensors as the multi-way arrays under a certain group 
action.}


\paragraph{Group actions on 3-way arrays.} 3-way arrays 
are simply arrays with 3 indices, generalizing the case of matrices (=2-way arrays). In this paper we consider entries of the arrays being from a field 
$\F$, so a 3-way array is just $\tA=(a_{i,j,k})$, $i\in[\ell]$, $j\in[n]$, 
$k\in[m]$, and $a_{i,j,k}\in\F$. 

Let $\GL(n, \F)$ be the general linear group of degree $n$ over $\F$, and let 
$\M(n,\F)$ denote the set of $n \times n$ matrices. There are three natural group 
actions on $\M(n,\F)$: for $A \in \M(n,\F)$, (1) $(P,Q) \in \GL(n,\F) \times 
\GL(n,\F)$ sends $A$ to $P^t A Q$, (2) $P \in \GL(n,\F)$ sends $A$ to $P^{-1} A 
P$, and (3) $P \in \GL(n,\F)$ sends $A$ to $P^t A P$.
These three actions 
then endow $A$ with different algebraic/geometric interpretations: (1) a linear map 
from a vector space $V$ to another vector space $W$, (2) a linear map from $V$ to 
itself, and (3) a bilinear map from $V\times V$ to $\F$. 

Likewise, 3-way arrays $\tA=(a_{i,j,k})$, $i, j, k\in[n]$, can be naturally acted 
by $\GL(n, \F)\times \GL(n, \F)\times \GL(n, \F)$ in one way, by $\GL(n, \F)\times 
\GL(n, \F)$ in two different ways, and by $\GL(n, \F)$ in two different ways. 
These five actions endow various families of 3-way arrays with 
different algebraic/geometric 
meanings, including 3-tensors, bilinear maps, matrix (associative or Lie) 
algebras, and trilinear forms (a.k.a. 
non-commutative cubic forms). (See \Sec{sec:problems} for detailed explanations.) 
Over finite fields, the associated isomorphism problems are in $\cc{NP}\cap\cc{coAM}$, following the essentially same 
$\cc{coAM}$ 
protocol as for \GI. 


With these group actions in mind, 3-way arrays capture a variety of important structures in several mathematical and computational disciplines. 
They arise naturally in quantum mechanics (states are described by 
tensors), the complexity of matrix multiplication (matrix multiplication is
described by a tensor, and its algebraic complexity is essentially its tensor 
rank), the Geometric Complexity Theory approach \cite{Mul11} to the Permanent versus 
Determinant Conjecture \cite{Val79} (tensors describe the boundary of the 
determinant orbit 
closure, e.\,g., \cite[Sec.~13.6.3]{Lan12} and
\cite[Sec.~3.5.1]{grochowPhD} for introductions,
and \cite{HL16, H17} for applications), data analysis \cite{KB09},  
machine learning 
\cite{PSS18}, computational group theory \cite{LQ17, BMW18}, and cryptography 
\cite{Pat96,JQSY19}.



\paragraph{Main results.} 
The five natural actions on 3-way arrays mentioned above each lead to a different 
isomorphism problem on 3-way arrays; we discuss these problems and their 
interpretations in \Sec{sec:problems}. 
Our first main result, \Thm{thm:main}, shows that these 
isomorphism problems for 3-way arrays are all equivalent under 
polynomial-time reductions. Due to the 
algebraic or geometric interpretations, these problems are further 
equivalent to isomorphism problems on certain classes of groups, 
cubic forms, trilinear forms (a.k.a. non-commutative cubic forms), 
associative algebras, and Lie algebras. 
One consequence of these results (\refcorp), along with those of \cite{FGS19}, is a reduction 
from \GpI for $p$-groups of exponent $p$ and class $< p$ to \GpI for $p$-groups of 
exponent $p$ and class 2. Although the latter have long been believed to be the 
hardest cases of \GpI, as far as we are aware, this is the first reduction from a 
more general class of groups to this class.

Although these equivalences may have been expected by some experts, it had not been 
immediately clear to us for some time during this project. 
To get a sense for the non-obviousness, let us postulate  the following hypothetical question. 
Recall that two matrices $A, B\in \M(n, \F)$ are called \emph{equivalent} if 
there exists $P, Q\in\GL(n, \F)$ such that $P^{-1}AQ=B$, and they are \emph{conjugate} if 
there 
exists $P\in \GL(n,\F)$ such that $P^{-1}AP=B$. Can we reduce testing \algprobm{Matrix Conjugacy}
to testing \algprobm{Matrix Equivalence}? Of course since they are both in 
$\cc{P}$ there is a trivial reduction; to avoid this, let us consider only 
reductions $r$ which send a matrix $A$ to a matrix $r(A)$ such that $A$ and $B$ 
are conjugate iff $r(A)$ and $r(B)$ are equivalent. Nearly all reductions between 
isomorphism problems that we are aware of have this form (so-called ``kernel 
reductions'' \cite{FortnowGrochowPEq}; 
cf. functorial reductions \cite{BabaiSR}). 
After some thought, we realize that this is essentially 
impossible. The reason is that the equivalence class of a matrix is completely determined by its 
rank, while the conjugacy class of a matrix is determined by its rational canonical form. Among $n \times n$ matrices there are only $n+1$ equivalence classes, but there are at least $|\F|^n$ rational canonical forms (coming from the choice of minimal polynomial/companion matrix). Even when $\F$ is a finite field, such a reduction would thus require an exponential increase in dimension, and when $\F$ is infinite, such a reduction is impossible (regardless of running time).

Nonetheless, one of our results is that for \emph{spaces} of matrices (one form of 3-way arrays), conjugacy testing does indeed reduce to equivalence testing!
This is in sharp contrast to the case of single matrices. In 
the above setting, it means that there exists a polynomial-time computable map 
$\phi$ from $\M(n, \F)$ to \emph{subspaces of} $\M(s, \F)$, such that $A, B$ are 
conjugate up to a scalar if and only if $\phi(A), \phi(B)\leq \M(s, \F)$ are 
equivalent as matrix spaces. Such a reduction may not be clear 
at first sight.

Our second main result reduces \DeeTI to \ThreeTI, for any fixed $d \geq 3$. 
From one viewpoint, this can be seen as a linear algebraic analogue of the 
now-classical reduction from $d$-uniform \algprobm{Hypergraph Isomorphism} to \GI (e.\,g., \cite{ZKT}). 
However, as the reader will see, the reduction here is quite a bit more involved, 
using quiver algebras and the Wedderburn--Mal'cev Theorem on complements of the 
Jacobson radical in associative algebras. From another viewpoint, this can be seen 
as a step towards showing that \ThreeTI is not only universal among isomorphism 
problems on 3-way arrays \cite{FGS19}, but perhaps \ThreeTI is already universal 
for isomorphism problems on $d$-way arrays for any $d$; see \Sec{sec:universality}.
These first two results indicate the robustness and naturality of the notion of 
$\cc{TI}$-completeness.

Our next set of results reduce \GIlong and \algprobm{Linear Code Equivalence} to these 
isomorphism problems for 3-way arrays (\Sec{sec:GI_code}). This 
shows that these isomorphism problems for 3-way arrays form a set of potentially 
harder problems than these two problems, as 
also supported by the current difference in their practical 
difficulties.\footnote{There is a heuristic 
algorithm for \algprobm{Linear Code Equivalence} by Sendrier \cite{Sen00}, which 
is practically effective in many cases, 
though for self-dual codes it reverts to an exponential search.}  It currently 
seems unlikely to us that either 
\GIlong or \CodeEqlong is $\cc{TI}$-complete. 

Finally, our third main contribution is to 
show a search-to-decision reduction for these 
tensor problems (\Thm{thm:search_decision}), which may be of independent interest,  leveraging our technique from above.
While such a reduction has long been known for \GI, for \GpIlong in general this 
remains a long-standing open question. Our techniques allow us to give a  
search-to-decision reduction for isomorphism of $p$-groups of class 2 and exponent $p$ in time 
$|G|^{O(\log \log |G|)}$ in the model of matrix groups over finite fields.
This group class is widely regarded to be the hardest cases of 
\GpIlong. As far as we know, this is the first 
non-trivial search-to-decision reduction for testing isomorphism of a class of finite groups. 

\paragraph{Implications of main results for practical 
computations.} Our first 
main result may partly help to explain the difficulties from various areas when dealing 
with these isomorphism problems. There is currently a significant difference between isomorphism 
problems for 3-way arrays and that for graphs. Namely, in sharp contrast to 
\GIlong---for 
which very effective practical algorithms have existed for some
time \cite{McK80,MP14}---the 
problems we consider here all still pose great difficulty even on relatively small 
examples in practice. Indeed, such problems have been proposed to be difficult 
enough 
for cryptographic purposes \cite{Pat96,JQSY19}. As further evidence of their 
practical 
difficulty, current 
algorithms implemented for \AltMatSpIsomlong\footnote{An $n\times n$ matrix $A$ 
over $\F$ is alternating if for every $v\in \F^n$, $v^tAv=0$. When $\F$ is not of 
characteristic $2$, this is equivalent to the skew-symmetry condition.}---a 
problem we show is $\cc{TI}$-complete---can handle the 
cases when the 3-way array is of size $10\times 10\times 10$ over $\F_{13}$, but 
absolutely  not for 3-way arrays of size $100\times 100\times 100$, even though in 
this case the input can still be stored  in only a 
few megabytes.\footnote{We thank James B. Wilson, who maintains a suite of 
algorithms for $p$-group isomorphism testing, for communicating this insight to us 
from his hands-on 
experience. We of course maintain responsibility for any possible 
misunderstanding, or lack of knowledge regarding the performance of other 
implemented algorithms.}
In \cite{PSS18}, motivated by machine learning applications, 
computations on one $\cc{TI}$-complete problem were performed in Macaulay2 \cite{M2}, but these could 
not go beyond small examples either. 
Our results imply that the complexities of these problems arising in many fields%
---from computational group theory to cryptography to machine learning---are all 
equivalent.

\paragraph{Isomorphism problems for 3-way arrays as a bottleneck for graph 
isomorphism.}
In addition to their many 
incarnations and practical uses mentioned above, the isomorphism problems we 
consider on 3-way arrays can be further motivated by their relationship to \GI. 
Specifically, these problems both form a key bottleneck to putting \GI into 
$\cc{P}$, and pose a great challenge for extending techniques used to solve \GI.

Isomorphism problems for 3-way arrays 
stand as a key bottleneck to put \GI in $\cc{P}$. 
This is because, as Babai 
pointed out 
\cite{Bab16}, \GpIlong is a key bottleneck to putting \GI into $\cc{P}$. 
Indeed, the current-best upper bounds on these two problems are now quite 
close: $n^{O(\log n)}$ for \GpIlong (originally due to \cite{FN70, 
Mil78}\footnote{Miller attributes this to 
Tarjan.}, with improved constants   
\cite{Wil14, Ros13a, Ros13b}), 
and 
$n^{O(\log^2 n)}$ for \GI \cite{Bab16} (see \cite{HBD17} for calculation of the exponent). 
Within \GpIlong, it is 
widely regarded, for several reasons 
(e.\,g., \cite{Bae38, HigmanEnum, SergeichukPgpWild, WilsonWildSlides}), that the 
bottleneck is the class of $p$-groups 
of class 2 and exponent $p$ (i.e., $G/Z(G)$ is abelian and $g^p=1$ for all $g$, 
$p$ odd). 
Then 3-way arrays enter the picture by Baer's Correspondence 
\cite{Bae38}, which shows that the isomorphism problem for these groups is 
equivalent to 
telling whether two linear spaces of skew-symmetric matrices over $\F_p$ are 
equivalent up to transformations of the form $A \mapsto P^t A P$. This is the 
\AltMatSpIsomlong problem, which we show in this paper is 
$\cc{TI}$-complete.\footnote{Because of the difference in verbosity of inputs, 
solving \GpIlong for this class of groups in time $\poly(|G|)$ 
is equivalent  to solving \AltMatSpIsomlong in time $p^{O(n+m)}$ for $n\times n$ matrix 
spaces of dimension $m$ over $\F_p$. The current state of the art is 
$p^{O(n^2)}$, which corresponds to the nearly-trivial upper bound of $|G|^{O(\log 
|G|)}$ on \GpIlong.}

To see why the techniques for \GI face great difficulty when dealing with 
isomorphism problems for multi-way arrays, recall that
most algorithms for \GI, including Babai's  
\cite{Bab16}, 
are built on 
two families of techniques: group-theoretic, and combinatorial. 
One of the main differences is 
that the underlying group 
action for \GI is a permutation group acting on a combinatorial structure, whereas 
the underlying group actions for isomorphism problems for 3-way arrays are matrix 
groups acting on (multi)linear structures. 

Already in moving from permutation groups to matrix groups, we find many new
computational difficulties 
 that arise naturally in basic subroutines used in isomorphism testing. For 
 example, the membership problem for 
permutation groups is well-known to be efficiently solvable by Sims's algorithm 
\cite{Sim78} (see, e.\,g., \cite{Ser03} for a textbook treatment),
while for matrix groups this was only recently 
shown to be 
solvable with a number-theoretic 
oracle over finite fields of odd characteristic 
\cite{BBS09}. 
Correspondingly, 
when moving from combinatorial structures to (multi)linear 
algebraic structures, we also find severe limitation on 
the use of most combinatorial techniques, like individualizing a vertex. For example,  
it is quite expensive to 
enumerate all 
vectors in a vector space, while it is usually considered efficient to go through 
all 
elements in a set. 
Similarly, within a set, any subset has a unique complement, whereas within 
$\F_q^n$, a subspace can have up to $q^{\Theta(n^2)}$ complements. 

Given all the differences between the combinatorial and linear-algebraic 
worlds, it may be surprising that combinatorial techniques for \GIlong can nonetheless be useful for \GpIlong. Indeed, guided by the 
postulate that alternating matrix spaces can be viewed as a linear algebraic 
analogue of graphs, Li and the second author \cite{LQ17} adapted the individualisation and 
refinement technique, as used by Babai, Erd\H{o}s and Selkow \cite{BES80}, to 
tackle \AltMatSpIsomlong over $\F_q$. This algorithm was recently 
improved \cite{BGL+19}. However, this technique, though helpful to improve from 
the brute-force
$q^{n^2}\cdot \poly(n, \log q)$ time, seems still limited to getting $q^{O(n)}$-time algorithms.


\paragraph{New techniques.} 
Our first new 
technique for the above results on 3-way arrays is to develop a linear-algebraic analogue 
of the coloring gadget used in the 
context of \GIlong (see, e.\,g., \cite{KST93}). 
These gadgets
help us to restrict to various subgroups of the general linear group. 
Recall that, in relating \GI with other isomorphism problems, coloring is a very 
useful idea. Given a graph $G=(V, E)$, a coloring of vertices is 
 a function $c:V \to C$ where $C$ is a set of ``colors.'' Colored isomorphism between two 
vertex-colored graphs asks only for isomorphisms that send vertices of one color 
to vertices of that same color.
If we are interested in 
making a specific vertex $v\in V$ special (``individualizing'' that vertex), we can assign this 
vertex a unique color. To reduce \algprobm{Colored Graph Isomorphism} to 
ordinary \GIlong uses certain gadgets, and we adapt this idea to the context of 
3-way arrays.  We note that 
\cite{FGS19} construct a related such gadget. In this paper, we develop a new gadget which we use both by itself, and in combination with the gadget from \cite{FGS19} (albeit in a new context), 
see \Sec{sec:related} and \Sec{sec:reduction_gadget}.

Our second new technique, used to show the reduction from \DeeTI to \ThreeTI, is a 
simultaneous generalization of our reduction from \ThreeTI to \AlgIsolong and the 
technique Grigoriev used \cite{Grigoriev83} to show that isomorphism in a certain 
restricted class of algebras is equivalent to \GI. In brief 
outline: a 3-way 
array $\tA$ specifies the structure constants of an algebra with basis $x_1, 
\dotsc, x_n$ via $x_i \cdot x_j := \sum_{k} \tA(i,j,k) x_k$, and this is 
essentially how we use it in the reduction from \ThreeTI to \AlgIsolong. For 
arbitrary $d \geq 3$, we would like to similarly use a $d$-way array $\tA$ to 
specify how $d$-tuples of elements in some algebra $\cA$ multiply. The issue is 
that for $\cA$ to be an algebra, our construction must still specify how 
\emph{pairs} of elements multiply. The basic idea is to let pairs (and triples, 
and so on, up to $(d-2)$-tuples) multiply ``freely'' (that is, without additional 
relations), and then to use $\tA$ to rewrite any product of $d-1$ generators as a 
linear combination of the original generators. While this construction as 
described already gives one direction of the reduction (if $\tA \cong \tB$, then 
$\cA \cong \cB$), the other direction is trickier. For that, we modify the 
construction to an algebra in which short products (less than $d-2$ generators) do 
not quite multiply freely, but almost. After the fact, we found out that this 
construction generalizes the one used by Grigoriev \cite{Grigoriev83} to show that 
\GI was equivalent \AlgIsolong for a certain class of algebras (see 
\Sec{sec:related} for a comparison).

\paragraph{Organization.} We aim to reach as wide an audience as 
possible, so we start with a detailed introduction to the 
various isomorphism problems on 
3-way arrays, and their algebraic and geometric 
interpretations in \Sec{sec:problems}. We then describe our results in 
detail in \Sec{sec:result} and consider related work in \Sec{sec:related}. An illustration of the key technique is in 
\Sec{sec:technique}. These sections may be viewed as 
an extended abstract. 

The remainder of the paper gives detailed proofs of all results. \Sec{sec:prel} contains additional
preliminaries. In \Sec{sec:reduction_gadget}, we present those 
reductions which use the linear-algebraic coloring technique, thus proving
\Thm{thm:main}(\ref{thm:main:isom}) and 
\Thm{thm:search_decision}. We then finish the proof of 
\Thm{thm:main} 
by presenting the remaining reductions in \Sec{sec:reduction_other}. 
\Thm{thm:d_to_3} is proved in \Sec{sec:dto3}.
In \Sec{sec:conclusion}, we put forward a theory of 
universality for basis-explicit linear structures, in analogy with \cite{ZKT}. 
While not yet complete, this seems to provide another justification for studying 
\TIlong and related problems, and it motivates some interesting open 
questions. In Appendix~\ref{app:cubic} we give a reduction from \CubicFormlong to \DFormlong for any $d \geq 3$ (for $d > 6$ this is easy; for $d=4$ it requires some work).

\section{Preliminaries: Group actions on 3-way arrays} 
\label{sec:problems}
The formulas for most natural group actions on 3-way arrays are somewhat unwieldy; our experience suggests that they are more easily digested when presented in the context of some of the natural interpretations of 3-way arrays as mathematical objects. To connect the interpretations with the formulas themselves, one technical tool is very useful, namely, given a 3-way array $\tA(i,j,k)$, we define its \emph{frontal slices} to be the matrices $A_k$ defined by $A_k(i,j) := \tA(i,j,k)$; that is, we think of the box of $\tA$ as arranged so that the $i$ and $j$ axes lie in the page, while the $k$-axis is perpendicular to the page. Similarly, its \emph{lateral slices} (viewing the 3D box of $\tA$ ``from the side'') are defined by $L_j(i,k) := \tA(i,j,k)$. An $\ell \times n \times m$ 3-way array thus has $m$ frontal slices and $n$ lateral slices.

A natural action on arrays of size $\ell \times n \times m$ is that of $\GL(\ell, \F) \times \GL(n,\F) \times \GL(m,\F)$ by change of 
basis in each of the 3 directions, namely $((P,Q,R) \cdot \tA)(i',j',k') = 
\sum_{i,j,k} \tA(i,j,k) P_{ii'} Q_{jj'} R_{kk'}$. We will see several 
interpretations of this action below.

\paragraph{3-tensors.} A 3-way array $\tA(i,j,k)$, 
where $i\in[\ell]$, 
$j\in[n]$, and $k\in[m]$, is naturally identified as a vector in 
$\F^\ell\otimes\F^n\otimes\F^m$. Letting $\vec{e_i}$ denote the $i$th 
standard basis vector of $\F^n$,  a standard basis of 
$\F^\ell\otimes\F^n\otimes\F^m$ is 
$\{\vec{e_i}\otimes\vec{e_j}\otimes\vec{e_k}\}$. Then $\tA$ represents the vector 
$\sum_{i,j,k}\tA(i,j,k)\vec{e_i}\otimes\vec{e_j}\otimes\vec{e_j}$ in 
$\F^\ell\otimes\F^n\otimes\F^m$. The natural action by 
$\GL(\ell, \F)\times\GL(n, \F)\times\GL(m, \F)$ above corresponds to changes of 
basis of the three vector spaces in the tensor product. The problem of deciding 
whether 
two 3-way arrays are the same under this action is called 
\ThreeTIlong.\footnote{Some authors call this \algprobm{Tensor Equivalence}; we use 
``\algprobm{Isomorphism}'' both because this is the natural notion of isomorphism 
for such objects, and because we will be considering many different equivalence 
relations on essentially the same underlying objects.}


\paragraph{Matrix spaces.} 
Given a 3-way array $\tA$, it is natural to consider the linear span of its frontal slices, $\cA = \langle A_1, \dotsc, A_m \rangle$, also called a \emph{matrix space}. 
One convenience of this viewpoint is that the action of $\GL(m,\F)$ becomes implicit: it corresponds to 
change of basis \emph{within} the matrix space $\cA$. 
This allows us to generalize the three natural equivalence relations on matrices to matrix 
spaces: (1) two $\ell \times n$ matrix spaces $\cA$ 
and $\cB$ are \emph{equivalent} if there exists $(P, Q) \in \GL(\ell, \F) \times 
\GL(n, \F)$ such 
that $P\cA Q = \cB$, where $P\cA Q := \{PA Q : A \in \cA\}$; (2) two 
$n \times n$ matrix spaces $\cA, \cB$ are \emph{conjugate} if there exists $P \in 
\GL(n, \F)$ such that $P \cA P^{-1} = \cB$; and (3) they are \emph{isometric} if 
$P \cA 
P^t = \cB$. The corresponding decision problems, when $\cA$ is 
given by a basis $A_1, \dotsc, A_d$, are \MatSpEquivlong, \MatSpConjlong, 
and \MatSpIsomlong, respectively. 

\hyperdef{}{sec:problems:nilpotent}{}
\paragraph{Nilpotent groups.} If $A,B$ are two subsets of a group $G$, then 
$[A,B]$ 
denotes the sub\emph{group} generated by all elements of the form $[a,b] = 
aba^{-1}b^{-1}$, for $a \in A, b \in B$. The \emph{lower central series} of a 
group $G$ is defined as follows: $\gamma_1(G) = G$, $\gamma_{k+1}(G) = 
[\gamma_k(G), G]$. A group is \emph{nilpotent} if there is some $c$ such that 
$\gamma_{c+1}(G) = 1$; the smallest such $c$ is called the \emph{nilpotency class} 
of $G$, or sometimes just ``class'' when it is understood from context. A finite 
group is nilpotent if and only if it is the product of its Sylow subgroups; in 
particular, all groups of prime power order are nilpotent.

\paragraph{Bilinear maps, finite groups, and systems of polynomials.} 
While the 
matrix 
space viewpoint has the merit of 
drawing an analogy with the more familiar object of matrices, other interpretations 
lead to standard complexity problems that may be more familiar to some readers. 
For example, from an $\ell \times n \times m$ 3-way array $\tA$, we can construct 
a bilinear map (=system of $m$ bilinear forms) $f_\tA:\F^\ell\times\F^n\to\F^m$, 
sending $(u, v)\in \F^\ell\times 
\F^n$ to $(u^t A_1 v, \dots, u^tA_m v)^t$, where the $A_k$ are the frontal slices of $\tA$.\footnote{In this paper elements in $\F^n$ are column vectors.} 
The group action defining \MatSpEquivlong 
is equivalent to the action 
of $\GL(\ell, \F)\times\GL(n, \F)\times \GL(m, \F)$ on such bilinear maps.  

When $\ell=n$, the action in \MatSpIsomlong is equivalent to the natural action of 
$\GL(n, \F)\times \GL(m, \F)$ on such bilinear maps.
Two bilinear maps that are essentially the same up to such basis changes 
are sometimes called pseudo-isometric \cite{BW12}.

Bilinear maps of the form $V\times V\to W$ turn out to arise naturally in group theory and 
algebraic geometry. When $A_k$ are skew-symmetric over $\F_p$, $p$ an 
odd prime, Baer's correspondence \cite{Bae38} gives a bijection between finite 
$p$-groups 
of class 2 
and exponent $p$, that is, in which $g^p = 1$ for all $g$ and in which $[G, G] 
\leq Z(G)$, and their corresponding bilinear maps $G/Z(G) \times G/Z(G) \to 
[G,G]$, given by $(gZ(G), hZ(G)) \mapsto [g,h]=ghg^{-1}h^{-1}$. Two such groups 
are isomorphic if and only if their corresponding bilinear maps are 
pseudo-isometric, if and only if, using the matrix space terminology, the 
matrix spaces they span are isometric. When $A_k$ are symmetric, by the classical 
correspondences between symmetric matrices and homogeneous quadratic forms, a 
symmetric bilinear map naturally yields a quadratic map from $\F^n$ to $\F^m$. The 
two quadratic maps are isomorphic if and only if the corresponding bilinear 
maps are pseudo-isometric.

\paragraph{Cubic forms \& trilinear forms.}
From a 3-way array 
$\tA$ 
we  can also construct a cubic form (=homogeneous degree 3 polynomial) $\sum_{i,j,k} 
\tA(i,j,k) x_i x_j x_k$, where $x_i$ are formal variables. 
If we consider the variables as commuting---or, equivalently, if $\tA$ is 
symmetric, meaning it is unchanged by permuting its three indices---we get an 
ordinary cubic form; if we consider them as non-commuting, we get a trilinear form 
(or ``non-commutative cubic form''). In either case, 
the natural notion of isomorphism here comes from the 
action of $\GL(n,\F)$ on the $n$ variables $x_i$, in which $P \in \GL(n,\F)$ transforms 
the preceding form into $\sum_{ijk} \tA(i,j,k) (\sum_{i'} P_{ii'} x_{i'})(\sum_{j'} 
P_{jj'} x_{j'})(\sum_{k'} P_{kk'} x_{k'})$. In terms of 3-way arrays, we get $(P 
\cdot \tA)(i', j', k') = \sum_{ijk} \tA(i,j,k) P_{ii'} P_{jj'} P_{kk'}$. The 
corresponding isomorphism 
problems are called \CubicFormlong (in the commutative case) and \NcCubicFormlong.


\paragraph{Algebras.} We may also consider a 3-way array 
$\tA(i,j,k)$, $i, j, 
k\in[n]$, as the structure 
constants of an algebra (which need not be associative, commutative, nor unital), 
say with basis $x_1, \dotsc, x_n$, and with multiplication given by $x_i \cdot x_j 
= \sum_k \tA(i,j,k) x_k$, and then extended (bi)linearly. Here the natural notion 
equivalence comes from the action of $\GL(n,\F)$ by change of basis on the $x_i$. 
Despite the seeming similarity of this action to that on cubic forms, it turns out 
to be quite different: given $P \in \GL(n,\F)$, let $\vec{x}' = P\vec{x}$; then we 
have $x_i' \cdot x_j' = (\sum_{i} P_{i' i} x_i)\cdot (\sum_{j} P_{j' j} x_j) 
 = \sum_{i,j} P_{i' i} P_{j' j} x_i \cdot x_j$ 
 $= \sum_{i,j,k} P_{i' i} P_{j' j} \tA(i,j,k) x_k = \sum_{i,j,k} P_{i' i} P_{j' j} 
 \tA(i,j,k) \sum_{k'} (P^{-1})_{kk'} x_{k'}$.
Thus $\tA$ becomes $(P \cdot \tA)(i',j',k') = \sum_{ijk} \tA(i,j,k) P_{i' i} P_{j' j} 
(P^{-1})_{k k'}$. The inverse in the third factor here is the crucial difference 
between this case and that of cubic or trilinear forms above, 
similar to the difference between matrix conjugacy and matrix isometry. The 
corresponding isomorphism problem is called \AlgIsolong.

\paragraph{Summary.} 
The isomorphism problems of the above structures all have 3-way arrays as the 
underlying object, but are determined by different group actions. It is not hard to 
see that there are 
essentially five group actions in total: \ThreeTIlong, \MatSpConjlong, 
\MatSpIsomlong, \NcCubicFormlong, and \algprobm{Algebra} \algprobm{Isomorphism}.
It turns out that these cover all the natural isomorphism problems on 3-way arrays 
in which the group acting is a product of $\GL(n,\F)$ (where $n$ is the side 
length of the arrays); see 
\Sec{sec:prelim:tensor} for discussion.


\section{Main results}\label{sec:result}

\subsection{Equivalence of isomorphism problems for 3-way arrays} 

\begin{definition}[{$d\cc{TI}, \cc{TI}$}]
For any field $\F$, $d\cc{TI}_\F$ denotes the class of problems that are 
polynomial-time Turing (Cook) reducible to \DeeTIlong over 
$\F$.\footnote{We follow a natural convention: when $\F$ is finite, a fixed 
algebraic extension of a finite field such as $\overline{\F}_p$, the rationals, or 
a fixed algebraic extension of the rationals such as $\overline{\Q}$, we consider 
the usual model of Turing machines; when $\F$ is $\mathbb{R}$, $\mathbb{C}$, the 
$p$-adic rationals $\Q_p$, or other more ``exotic'' fields, we consider this in 
the Blum--Shub--Smale model over $\F$.} When we write $d\cc{TI}$ without 
mentioning the field, the result holds for any field. 
$\cc{TI}_{\F} = \bigcup_{d \geq 1} d\cc{TI}_{\F}$. 
\end{definition}

We now state our first main theorem. 
\begin{maintheorem}\label{thm:main}
\ThreeTIlong reduces to each of the following problems in polynomial time.

\begin{enumerate}
\item \GpIlong for $p$-groups exponent $p$ ($g^p=1$ for all $g$) and class 2 
($G/Z(G)$ is abelian) given by generating matrices over $\F_{p^e}$. Here we consider only $\cc{3TI}_{\F_{p^e}}$ where $p$ is 
an odd prime.

\item \label{thm:main:isom} %$V \otimes V \otimes W$ 
\MatSpIsomlong, even for alternating or symmetric matrix spaces.

\item %$V \otimes V^* \otimes W$ 
\MatSpConjlong, and even the special cases: 
\begin{enumerate}
\item \algprobm{Matrix Lie Algebra Conjugacy}, for solvable Lie algebras $L$ of 
derived length 2.\footnote{And even further,  where $L/ [L, L] \cong \F$.} % to 
%save a line
\item \algprobm{Associative Matrix Algebra Conjugacy}.\footnote{Even for algebras 
$A$ whose Jacobson radical $J(A)$ squares to zero and $A/J(A) \cong \F$.} % to 
%save a line
\end{enumerate}

\item %$V \otimes V \otimes V^*$ 
\algprobm{Algebra Isomorphism}, and even the special cases:
\begin{enumerate}
\item \algprobm{Associative Algebra Isomorphism}, for algebras that are 
commutative and unital, and for algebras that are commutative and 3-nilpotent 
($abc=0$ for all $a,b,c, \in A$)
\item \algprobm{Lie Algebra Isomorphism}, for 2-step nilpotent Lie algebras ($[u,[v,w]] = 0$ $\forall u,v,w$) % to save a line
\end{enumerate}
\item %$V \otimes V \otimes V$ 
\CubicFormlong and \NcCubicFormlong.
\end{enumerate}
The algebras in (3) are given by a set of matrices which linearly span the 
algebra, while in (4) they are given by structure constants (see ``Algebras'' in 
\Sec{sec:problems}).
\end{maintheorem}

\begin{remark}
Agrawal \& Saxena \cite[Thm.~5]{AS05} gave a reduction from \CubicFormlong over $\F$ to \algprobm{Ring Isomorphism} for commutative, unital, associative algebras over $\F$, when every element of $\F$ has a cube root. 
For finite fields $\F_q$, the 
only such fields are those for which $q=p^{2e+1}$ and $p\equiv 2 \pmod{3}$, which 
is asymptotically half of all primes. As explained after the proof of 
\cite[Thm.~5]{AS05}, the use of cube roots
seems inherent in their reduction.

Using our results in conjunction with \cite{FGS19}, we get a new reduction from 
\CubicFormlong to \algprobm{Ring Isomorphism} 
(for the same class of rings) which works over 
any field of characteristic 0 or $p > 3$. 
Note that our reduction is very different from the one in 
\cite{AS05}.
\end{remark}

Figure~\ref{fig:main} below summarizes where the various parts of \Thm{thm:main} are proven.

We then resolve an open question well-known to the experts:\footnote{We asked several experts who knew of the question, but we were unable to find a written reference. Interestingly, Oldenburger \cite{oldenburger} worked on what we would call \DeeTIlong as far back as the 1930s. We would be grateful for any prior written reference to the question of whether \DeeTI reduces to \ThreeTI.}

\setcounter{maincorollary}{\arabic{maintheorem}}
\begin{maintheorem} \label{thm:d_to_3}
\DeeTIlong reduces to \AlgIsolong.
\end{maintheorem}

Since the main result of \cite{FGS19} reduces the problems in Theorem~\ref{thm:main} to 
\ThreeTIlong (cf. \cite[Rmk.~1.1]{FGS19}), we have:

\begin{maincorollary} \label{cor:main}
Each of the problems listed in Theorem~\ref{thm:main} is $\cc{TI}$-complete.\footnote{For \CubicFormlong, we only show that it is in $\cc{TI}_\F$ when $\chr 
 \F > 3$ or $\chr \F = 0$.}
In particular, $d\TI$ and $\ThreeTI$ are equivalent.
\end{maincorollary}

\begin{remark}
This phenomenon is reminiscent of the transition in hardness from 2 to 3 in 
$k$-\algprobm{SAT}, $k$-\algprobm{Coloring}, $k$-\algprobm{Matching}, and many 
other $\cc{NP}$-complete problems. It is interesting that an analogous 
phenomenon---a transition to some sort of ``universality'' from 2 to 3---occurs in 
the setting of isomorphism problems, which we believe are not $\cc{NP}$-complete 
 over finite fields.
\end{remark}

\newif\iffgs
\fgsfalse

\begin{figure}[!htbp]
\[
\hspace{-0.35in}
\xymatrix{
 & & \parbox{1in}{\centering $d$\algprobm{-Tensor Iso.} \\ $U_1 \otimes \dotsb \otimes U_d$} \ar@/^1pc/[dddr]^(0.7){\text{\Thm{thm:d_to_3}}}\\
 & & \ar@/_/[dll]_{\parbox{0.5in}{\scriptsize \text{\Prop{prop:3-tensor_isometry},} \\ \text{\Cor{cor:pgp}}}} \parbox{1in}{\centering \algprobm{3-Tensor Iso.} \\ $U \otimes V \otimes W$}\ar@/^/[drr]^{\text{Prop.~\ref{prop:3-tensor_conjugacy}}} & & \\
\parbox{1.3in}{\centering \algprobm{Bilinear Map Iso.} \\ p-\algprobm{Group Iso.} \\ $V \otimes V \otimes W$} \iffgs \ar@/_/[urr]_{\text{\cite{FGS19}}}\fi \ar[dr]_{\text{Prop.~\ref{prop:isometry_algebra}}} \ar@/^/[drrr]^(.65){\text{Prop.~\ref{prop:isometry_algebra}}} \ar@/^1.4pc/[ddrrr]^(.65){\text{\Cor{cor:pseudo_special}}} & & & &  \iffgs \ar@/^/[ull]^{\text{\cite{FGS19}}}\fi \parbox{1in}{\centering \algprobm{Matrix Space} \\ \algprobm{Conjugacy} \\ $V \otimes V^* \otimes W$} \\
&  \parbox{1in}{\centering \algprobm{Trilinear} \\ \algprobm{Form} \algprobm{Equiv.} \\ $V \otimes V \otimes V$} \iffgs \ar[uur]_(.65){\text{\cite{FGS19}}}\fi & & \iffgs \ar[uul]_(.35){\text{\cite{FGS19}}}\fi \parbox{1in}{\centering \algprobm{Algebra Iso.} \\ $V \otimes V \otimes V^*$} &  \\
 & \parbox{1in}{\centering \algprobm{Cubic Form} \ar[u]^(.45){\text{Special case, when $6$ is a unit}} \\ \algprobm{Equiv.}} & & \ar[ll]^{\text{\cite{AS05, AS06}}} \ar[u]_{\text{Special case}} \parbox{1in}{\centering \algprobm{Commutative} \\ \algprobm{Algebra Iso.}}
 }
\]

\caption{\label{fig:main} Reductions for Thm.~\ref{thm:main}. An arrow $A \to B$ indicates that $A$ reduces to $B$, i.\,e., $A \leq_m^p B$. For \Cor{cor:main}, the five tensor problems in the center circle all reduce to \ThreeTI via \cite{FGS19}. For the ``$V \otimes V \otimes W$'' notation, see \Sec{sec:prelim:tensor}. } 
\end{figure}


\begin{remark}
Here is a brief summary of what is known about the complexity of some of these 
problems.
Over a finite field $\F_q$, these problems are in $\cc{NP}\cap \cc{coAM}$. For $\ell \times n \times m$ 3-way arrays, the brute-force algorithms run in time $q^{O(\ell^2 + n^2 + m^2)}$, 
as $\GL(n,\F_q)$ can be enumerated in time $q^{\Theta(n^2)}$. Note 
that polynomial-time in the input size here would be $\poly(\ell, n, m, \log q)$.
Over any field $\F$, these problems are in $\cc{NP}_{\F}$ in the Blum--Shub--Smale model. 
When the input arrays are over $\Q$ and we ask for isomorphism over $\C$ or 
$\mathbb{R}$, these problems are in $\cc{PSPACE}$ using quantifier elimination. By 
Koiran's celebrated result on Hilbert's Nullstellensatz, for equivalence over $\C$ 
they are in $\cc{AM}$ assuming the Generalized Riemann Hypothesis \cite{Koi96}. 
When the input is over $\Q$ and we ask for equivalence over $\Q$, it is unknown 
whether these problems are even decidable; classically this is studied under 
\AlgIsolong for associative, unital algebras over 
$\Q$ (see, e.\,g., \cite{AS06, Poonen}), but by 
\Cor{cor:main}, the question of decidability is open for all of these problems.

Over finite fields,
several of these problems can be solved efficiently when one of the side lengths 
of the array is small. For $d$-dimensional spaces of $n \times n$ matrices, \MatSpConjlong and \algprobm{Isometry}
can be solved in 
$q^{O(n^2)}\cdot \poly(d,n,\log q)$ time: once we fix an element of 
$\GL(n,\F_q)$, the isomorphism problem reduces to solving linear systems of equations.
Less trivially, \MatSpConjlongWords 
can be solved in time 
$q^{O(d^2)}\cdot \poly(d,n,\log q)$ and \ThreeTI for $n \times m \times d$ tensors 
in time $q^{O(d^2)}\cdot \poly(d,n,m,\log q)$, since once we fix an element of 
$\GL(d,\F_q)$, the isomorphism problem either becomes an instance of, or reduces to 
\cite{IQ17},  \algprobm{Module Isomorphism}, which admits several polynomial-time 
algorithms \cite{BL08, CIK97, IKS10, Sergeichuk2000}. Finally, one can solve 
\MatSpIsomlongWords in time $q^{O(d^2)}\cdot 
\poly(d,n, \log q)$: once one fixes an element 
of 
$\GL(d,\F_q)$, there is a rather involved algorithm  
\cite{IQ17}, which uses the $*$-algebra technique originated from the study of 
computing with 
$p$-groups \cite{Wil09a,BW12}.
\end{remark}

\subsection{Relations with Graph Isomorphism and Code 
Equivalence}\label{sec:GI_code}
We observe then \GIlong and \CodeEqlong reduce to \ThreeTIlong. In particular, the 
class $\cc{TI}$ contains the classical graph isomorphism class $\cc{GI}$. 

Recall \CodeEqlong asks to decide whether two linear codes are the 
same up to a linear transformation preserving the Hamming weights of codes. Here 
the linear codes are just subspaces of $\F_q^n$ of dimension $d$, represented by 
linear bases. Linear transformations preserving the Hamming weights include 
permutations and monomial transformations. Recall that the latter consists of matrices 
where every row and every column has exactly one non-zero entry. 
Indeed, over many fields this is without loss of generality, as Hamming-weight-preserving 
linear maps are always induced by monomial transformations (first proved over 
finite fields \cite{MacWilliams}, and more recently over much more general 
algebraic objects, e.\,g., \cite{GNW}). 
\CodeEq has long 
been studied in the coding theory community; see e.g. \cite{PR97,Sen00}.

For \CodeEqlong, we observe that previous results already combine to give:
\begin{observation}\label{obs:code_3TI}
\CodeEqlong (under permutations) reduces to \ThreeTIlong.
\end{observation}

\begin{proof}
 \CodeEqlong reduces to \MatLieConjlong \cite{GrochowLie}, a special case of 
 \MatSpConjlong, which in turn reduces to \ThreeTI \cite{FGS19}.
\end{proof}

Using the linear-algebraic coloring gadget, we can extend this to equivalence of 
codes under monomial transformations (see \Sec{sec:technique}). Given two 
$d\times n$ matrices $A, B$ over $\F$ of rank $d$, the \MonCodeEqlong
problem is to decide whether there exist $Q\in \GL(d, \F)$ and a monomial 
matrix $P\in \Mon(n, \F)\leq \GL(n, \F)$ 
(product of a diagonal matrix and a permutation matrix) such that $QAP =B$. 

\begin{proposition} \label{prop:MonCodeEq}
\MonCodeEqlong reduces to \ThreeTIlong.
\end{proposition}

Since \GIlong reduces to \CodeEqlong \cite{Luks} (see \cite{miyazakiCodeEq}) and \cite{PR97} (even over arbitrary fields \cite{GrochowLie}), by 
\Obs{obs:code_3TI} and \Thm{thm:main}, we have the following.
\begin{corollary}
\GIlong reduces to \AltMatSpIsomlong.
\end{corollary}

Using our linear-algebraic gadgets, we also
reprove this result using a much more 
direct reduction (see \Prop{prop:GI}). Besides being a different 
construction, another reason for the additional proof is that the technique leads to the 
search-to-decision reduction, which we discuss below.

\subsection{Application to \GpIlong: reducing the nilpotency class}
For several reasons, the hardest cases of \GpIlong are believed to be $p$-groups 
of class 2 and exponent $p$; recall that these are groups in which every element 
has order $p$, the order of the group is $p^n$, and $G/Z(G)$ is abelian. See 
\hyperref{}{}{sec:problems:nilpotent}{Nilpotent groups} above. While this belief  
has been widely held for many decades, we are not aware of any prior reduction 
from a more general 
class of groups to this class. However, by combining our results with the Lazard 
correspondence, we immediately get such a reduction.

\hyperdef{}{cor:p}{}
\begin{corp}
Let $p$ be an odd prime. 
For groups generated by $m$ matrices of size $n \times n$, \GpIlong for $p$-groups 
of exponent $p$ and class $c < p$ reduces to \GpIlong for $p$-groups of exponent 
$p$ and class $2$ in time $\poly(n, m, \log p)$.
\end{corp}

\begin{proof}
By the Lazard correspondence 
(reproduced as \Thm{thm:lazard} below) 
two $p$-groups of exponent $p$ and class $c < p$ are isomorphic if and only if 
their corresponding $\F_p$-Lie algebras are. 
By \Prop{prop:lazard_matrices}, we can construct a generating set for the 
corresponding Lie algebra by applying the power series for logarithm to the 
generating matrices of $G$. This Lie algebra is thus a subalgebra of $n \times n$ 
matrices, so we can generate the entire Lie algebra (using the linear-algebra 
version of breadth-first search; its dimension is $\leq n^2$) and compute its 
structure constants in time polynomial in $n$, $m$, and $\log p$. 
Then use \cite{FGS19} to reduce isomorphism of Lie 
algebras to \TI, and then apply \Thm{thm:main} (specifically, \Cor{cor:pgp}) to reduce to isomorphism of 
$p$-groups of exponent $p$ and class $2$ given by a matrix generating set.
\end{proof}

The only obstacle to getting this proof to work in the Cayley table model is that 
our reduction from \TI to \AltMatSpIsomlong (\Prop{prop:3-tensor_isometry}) blows 
up the dimension quadratically, which means the size of the group becomes 
$|G|^{O(\log |G|)}$ after the reduction. See 
Question~\ref{question:search_decision}. 



\subsection{Search to decision reductions} 
Reducing search problems to their associated decision problems 
is a classical and intriguing topic in complexity 
theory. Aside from the now-standard search-to-decision reduction for SAT, one of the earliest results in this direction was by Valiant in the 1970's \cite{valiant}. A celebrated 
result of Bellare and Goldwasser shows that, assuming $\cc{EE}\neq\cc{NEE}$, there exists a language in $\cc{NP}$ for 
which search does not reduce to decision under polynomial-time reductions 
\cite{BG94}. 
However, as usual for such statements based on complexity-theoretic assumptions, 
the problems constructed by such a proof are considered somewhat unnatural. For 
natural problems, on the one hand, there are search-to-decision reductions for 
$\cc{NP}$-complete problems and for \GI. On the other hand, such is not 
known, nor expected to be the case, for Nash Equilibrium  \cite{CDT09} (for which decision is 
trivial). 

Reducing search to decision is particularly intriguing for testing isomorphism 
of groups. One difficulty is that it is not clear how to 
guess a partial solution, and then make progress by restricting to a 
subgroup. In general, testing isomorphism 
of certain algebraic structures (groups, algebras, etc.) forms a large family of 
problems 
for which search-to-decision reductions are not known.

Because of the close relationship between \ThreeTI and isomorphism of various 
algebraic structures, one might expect similar difficulties in reducing search to 
decision for \ThreeTI, 
and thus for $\cc{TI}$-complete problems as well. 
Nonetheless, for \AltMatSpIsomlong, we are able to use the linear-algebraic 
coloring gadgets to get a non-trivial search-to-decision reduction.

\begin{maintheorem}\label{thm:search_decision}
There is a search-to-decision reduction for \AltMatSpIsomlong which, given 
$n \times n$ alternating matrix spaces $\cA, \cB$ over $\F_q$,  computes an isometry between them if they are 
isometric, in time $q^{\tilde{O}(n)}$. The reduction queries the decision oracle 
with inputs of dimension at most $O(n^2)$.
\end{maintheorem}

As a consequence, a $q^{\tilde{O}(\sqrt{n})}$-time decision algorithm would result 
in a $q^{\tilde{O}(n)}$-time search algorithm, in contrast with the brute-force 
$q^{O(n^2)}$ running time. Note that in this context, the size of the input is 
$\poly(n,\log q)$, so a $q^{\tilde{O}(\sqrt{n})}$ running time is still quite 
generous.

By the connection between \AltMatSpIsomlong and \GpIlong for $p$-groups of class $2$ and exponent $p$, we have 
the following. 
Note that the natural succinct input representation mentioned in the following 
result can have size $\poly(\ell, \log p) = \poly(\log |G|)$. 

\begin{maincorollary} \label{cor:search_decision}
Let $p$ be an odd prime, and let \algprobm{GpIso2Exp($p$)} denote the 
isomorphism problem for $p$-groups of class 2 and exponent $p$ in the model of 
matrix groups over $\F_p$.
For groups of order $p^\ell$, there is a search-to-decision reduction 
for \algprobm{GpIso2Exp($p$)} running in time $|G|^{O(\log 
\log |G|)}=p^{\tilde{O}(\ell)}$.
\end{maincorollary}

\section{Related work} \label{sec:related}
The most closely related work is that of Futorny, Grochow, and Sergeichuk 
\cite{FGS19}. They show that a large family of 
isomorphism problems on 3-way 
arrays---including those involving multiple 3-way arrays simultaneously, or 3-way 
arrays that are partitioned into blocks, or 3-way arrays where some of the blocks 
or sides are acted on by the same group (e.\,g., \MatSpIsomlong)---all reduce to 
\ThreeTI. Our work complements theirs in that all our reductions 
for \Thm{thm:main}
go in the 
opposite direction, reducing \ThreeTI to other problems. Some of our other results 
relate \GI and \CodeEqlong to \ThreeTI; the latter problems were not considered in 
\cite{FGS19}. 
\Thm{thm:d_to_3} considers $d$-tensors for any $d \geq 3$, which were not 
considered in \cite{FGS19}.

In \cite{AS05,AS06}, Agrawal and Saxena considered \algprobm{Cubic Form 
Equivalence} and testing isomorphism of commutative, associative, unital algebras.
They showed that \GI reduces to \AlgIsolong; \algprobm{Commutative Algebra Isomorphism} reduces to 
\CubicFormlong;
and \algprobm{Homogeneous} \algprobm{Degree-$d$} \algprobm{Form} \algprobm{Equivalence} reduces to 
\AlgIsolong
assuming that the underlying field has $d$th root for every field element. 
By combining a reduction from \cite{FGS19}, 
\Prop{prop:3-tensor_isometry}, and \Cor{cor:pseudo_special}, we get a new 
reduction from \CubicFormlong to \algprobm{Algebra Isomorphism} that works over 
any field in which $3!$ is a unit, which is fields of characteristic $0$ or $p > 3$.

There are several other works which consider related isomorphism problems. 
Grigorev \cite{Grigoriev83} showed that \GI is equivalent to 
isomorphism of unital, associative algebras $A$ such that the radical $R(A)$ 
squares to zero and $A/R(A)$ is abelian. Interestingly, we show 
$\cc{TI}$-completeness for conjugacy of 
matrix algebras with the same abstract structure (even when $A/R(A)$ is only 
1-dimensional). Note the latter problem is equivalent to asking whether two 
representations of $A$ are equivalent up to automorphisms of $A$. In the proof of \Thm{thm:d_to_3}, which uses algebras in which $R(A)^d=0$ when reducing from \DeeTI, we use Grigoriev's result.

Brooksbank and Wilson \cite{BW15} showed a reduction from \algprobm{Associative 
Algebra Isomorphism} (when given by structure constants) to \algprobm{Matrix 
Algebra Conjugacy}. 
Grochow \cite{GrochowLie}, among other things, showed that \GI and \CodeEq reduce to \MatLieConjlong, which is a special case of \MatSpConjlong.

In \cite{KS06}, Kayal and Saxena considered testing isomorphism of finite rings 
when the rings are given by structure constants. This problem 
generalizes testing isomorphism of algebras over finite fields. They 
put this problem in $\cc{NP} \cap \cc{coAM}$ \cite[Thm.~4.1]{KS06}, reduce \GI to 
this problem \cite[Thm.~4.4]{KS06}, and 
prove that counting the number of ring automorphism (\#RA) is in 
$\cc{FP}^{\cc{AM} 
\cap \cc{coAM}}$ \cite[Thm.~5.1]{KS06}. They also present a $\cc{ZPP}$ reduction 
from \GI to \#RA, 
and show that the decision version of the ring automorphism problem is in $\cc{P}$.


To summarize this zoo of isomorphism problems and reductions, we include Figure~\ref{fig:summary} for reference.

\begin{figure}[!htbp]
\newcommand{\fieldfootnote}{\hyperref{}{}{fn:field}{\textsuperscript{*}}}
\newcommand{\fieldfootnoteother}{\hyperref{}{}{fn:field2}{\textsuperscript{$\dagger$}}}
\newcommand{\ronyaifootnote}{\hyperref{}{}{fn:ronyai}{\textsuperscript{$\ddagger$}}}
\[
\hspace{-0.5in}
\xymatrix{
\text{\parbox{1.1in}{\centering \algprobm{Symmetric $d$-Tensor Diagonal Iso.}}} \ar@/^/[d] &  \text{\parbox{1.1in}{\centering \algprobm{Matrix $p$-Group Iso.} (class 2, exp. $p$)}}  \ar[d]_{\text{\cite{Bae38}}} & & & \text{$d$-\algprobm{Tensor Iso.}} \ar@/^/[dddl]^(0.3){\text{\Thm{thm:d_to_3}}} %\ar[lll] \text{\parbox{1in}{\centering \algprobm{Genus $2$ Matrix $p$-Group Iso} (class 2, exp. $p$)}} \ar@/^5pc/[dddddd]^(0.9){\text{\cite{BMW, LW12}}} 
\\
\text{\parbox{1.1in}{\centering \algprobm{Degree-$d$ Form Eq.}}} \ar[rd]^(0.35){\text{\cite{AS05}\fieldfootnote}} \ar@/^/[u]^{\fieldfootnoteother}  &\text{\parbox{1in}{\centering \algprobm{Alt. Mat. Space Isom.}}} \ar[u] \ar@/^/[r]^{\text{\cite{FGS19}}} \ar[d]^(0.6){\text{\Prop{prop:isometry_algebra}}} & \ar@/^/[l]^{\text{\Thm{thm:main}}} \text{\parbox{1in}{3-\algprobm{Tensor Iso.}}}  \ar@/^1.25pc/[urr]^{\text{\Obs{obs:d}}} \ar@/_/[r]_{\text{\Thm{thm:main}}}& \ar@/_/[l]_{\text{\cite{FGS19}}} \text{\parbox{1in}{\centering\algprobm{Mat. Space Conj.}}} \\
\text{\parbox{1in}{\centering \algprobm{Cubic Form Eq.}}} \ar[u]^{\text{\Prop{prop:cubic_to_d}}} \ar@/^/[r]^{\text{\parbox{1.0in}{\centering \Thm{thm:main}\fieldfootnoteother}}} \ar[ru]^(0.3){\text{\cite{FGS19}\fieldfootnoteother}} & \text{\parbox{1in}{\centering \algprobm{Ring Iso.} (basis)}} \ar@/^2pc/[ld] \ar[drr] \ar@/^/[l]^{\text{\cite{AS05,AS06}}} & & \text{\parbox{1in}{\centering \algprobm{Mat. Assoc. Alg. Conj.}}}\ar[u]  &\text{\parbox{1in}{\centering\algprobm{Mat. Lie Alg. Conj.}}} \ar[ul] \\
\text{\parbox{1in}{\centering \algprobm{Ring Iso.} (gens/rels)}} & & & \text{\parbox{1in}{\centering\algprobm{Unital Assoc. Alg. Iso.}}} \ar[u]^{\text{\cite{BW15}}} & \text{\parbox{1in}{\centering\algprobm{Comm. Mat. Lie Alg. Conj.}}} \ar[u] \\
& & \text{\parbox{1in}{\centering \algprobm{Mon. Code Eq.}}} \ar[uuu]_(0.71){\text{\Prop{prop:MonCodeEq}}} & \ar[ld] \text{\parbox{1in}{\centering\algprobm{Semisimple Mat. Lie Alg. Conj.}}} \ar[ruu] &  \ar[dl] \text{\parbox{1in}{\centering\algprobm{Diag. Mat. Lie Alg. Conj.}}} \ar[u]\\
\text{\parbox{1in}{\centering \algprobm{Factoring Integers}}} \ar[uu]_{\text{\cite{AS05}}} \ar@/^1pc/[rrruu]^(0.3){\text{\parbox{0.55in}{\centering \cite{Ronyai88} (over $\Q$)\ronyaifootnote}}}
& \text{\parbox{1in}{\centering \algprobm{String Isomorphism}}} & \ar[l]^(0.35){\text{\parbox{0.7in}{\centering \cite{Luks82}, cf. \cite{Luks}}}}\text{\GI} \ar[luuu]^(0.6){\text{\cite{AS05,KS06}}} \ar@/_2.5pc/[luuuu]_(0.7){\text{\Prop{prop:GI}}} \ar[u]_(0.6){\text{cf. \cite{GrochowLie}}} \ar[r]_(0.35){\text{\cite{PR97,Luks}}} \ar[ru]_(0.6){\text{\cite{GrochowLie}}}  & \text{\parbox{1in}{\centering\algprobm{Perm. Code Eq.}}} \ar[ru]_(0.6){\text{\cite{GrochowLie}}} \ar[r]_(0.4){\text{\cite{BCGQ}}}& \text{\parbox{1in}{\centering \algprobm{Perm. Group Conj.}}} \\
\text{\parbox{1in}{\centering \algprobm{Alt. Mat. Space Isom.} ($\F_{p^e}$, verbose)}} \ar[r]^{\text{\cite{Bae38}}} & \text{\parbox{1.1in}{\centering \algprobm{$p$-Group Iso.} (class 2, exp. $p$, table)}} \ar[l] \ar[r] & \text{\parbox{1in}{\centering \algprobm{Group Iso.} (table)}} \ar[u]_{\text{(Classical, cf. \cite{ZKT})}} & %\ar[l] \text{\parbox{1in}{\centering \algprobm{Semisimple Group Iso.} (table)}} \ar[r]^(0.65){\text{\cite{BCQ12}}} & \cc{P}
%3TI-complete box
\ar@{--}(18,10);(165,10)^(0.9){\text{\normalsize $\cc{TI}$-complete}} % top edge 
\ar@{--}(18,10);(18,-63) % left edge
\ar@{--}(18,-63);(165,-63) % bottom edge
\ar@{--}(165,10);(165,-63) % right edge
%3TI-complete over special fields box
\ar@{--}(18,10);(-15,10)_{\text{\normalsize $\cc{TI}$-complete\fieldfootnote\fieldfootnoteother}} % top edge
\ar@{--}(-15,10);(-15,-44) % left edge
\ar@{--}(-15,-44);(18,-44) % bottom edge
 }
\]


\caption[Summary of isomorphism problems around \GIlong and 
\ThreeTIlong.]{\label{fig:summary} 
Summary of isomorphism problems around \GIlong and 
\TIlong. $A \to B$ indicates that $A$ reduces to $B$, i.\,e., $A \leq_m^p B$. Unattributed arrows indicate $A$ is clearly a special case of $B$. Note that the 
definition of ring used in \cite{AS05} is commutative, finite, and unital; by 
``algebra'' we mean an algebra (not necessarily associative, let alone commutative 
nor unital) over a field. The reductions between \algprobm{Ring Iso.} (in the 
basis representation) and \algprobm{Degree-$d$ Form Eq.} and \algprobm{Unital 
Associative Algebra Isomorphism} are for rings over a field. The equivalences 
between \AltMatSpIsomlong and $p$-\GpIlong are for matrix spaces over $\F_{p^e}$. 
Some 
\TI-complete problems from \Thm{thm:main} are left 
out for clarity.

\hrulefill

\hyperdef{}{fn:field}{{\color{red}\textsuperscript{*}}} These results only hold over fields where every element has a $d$th root. In particular, 
\algprobm{Degree $d$ Form Equivalence} and \algprobm{Symmetric $d$-Tensor 
Isomorphism} are \ThreeTI-complete over fields with $d$-th roots. 
A finite field $\F_q$ has this property if and only if $d$ is coprime to 
$q-1$. 

\hyperdef{}{fn:filed2}{{\color{red}\textsuperscript{$\dagger$}}} These results only hold over rings where $d!$ is a unit.

\hyperdef{}{fn:ronyai}{{\color{red}\textsuperscript{$\ddagger$}}}Assuming the Generalized Riemann Hypothesis, R\'{o}nyai \cite{Ronyai88} shows a Las Vegas randomized polynomial-time reduction from factoring square-free integers---probably not much easier than the general case---to isomorphism of 4-dimensional algebras over $\Q$. Despite the additional hypotheses, this is notable as the target of the reduction is algebras of \emph{constant} dimension, in contrast to all other reductions in this figure.
 }
\end{figure}

\section{Overview of one new technique, and one full proof}\label{sec:technique}

In this section we describe one of the key new
techniques in this paper: a linear-algebraic 
coloring gadget. We exhibit this gadget by giving the full proof of 
\Prop{prop:MonCodeEq} as an example. A related gadget was used in 
\cite{FGS19} to show reductions \emph{to} \ThreeTI; our reductions all go in the 
opposite direction. Furthermore, whereas the gadgets used in \cite{FGS19} were 
primarily to ensure that two different blocks could not be mixed, our gadgets 
allow us to ensure that certain slices of a tensor can be permuted, while 
disallowing more general linear transformations. 

In the context of \GI, there are many ways to reduce \algprobm{Colored GI} to ordinary \GI; here we give one example, which will serve as an analogy for our linear-algebraic gadget. To individualize a vertex $v \in G$ (give it a unique color), attach to it a large ``star'': if $|V(G)|=n$, add $n+1$ new vertices to $G$ and attach them all to $v$; call the resulting graph $G_v$. This has the effect that any automorphism of $G_v$ must fix $v$, since $v$ has a degree strictly larger than any other vertex. Furthermore, if $H_w$ is obtained by a similar construction, then there is an isomorphism $G \to H$ which sends $v \mapsto w$ if and only if $G_v \cong H_w$. Finally, if we attach stars of size $n+1$ to multiple vertices $v_1, \dotsc, v_k$, then any automorphism of $G$ must permute the $v_i$ amongst themselves, and there is an isomorphism $G \to H$ sending $\{v_1, \dotsc, v_k\} \mapsto \{w_1, \dotsc, w_k \}$ if and only if the corresponding enlarged graphs are isomorphic.

We adapt this idea to the context of 3-way arrays. 
Let 
$\tA$ 
be an $\ell \times n \times m$ 3-way array, with lateral slices $L_1, L_2, \dotsc, 
L_n$ (each an $\ell \times m$ matrix). 
For any vector $v \in \F^n$, we get an associated lateral matrix $L_v$, which is a linear combination of the lateral slices as given, namely $L_v := \sum_{j=1}^n v_j L_j$ (note that when $v=\vec{e_j}$ is the $j$-th standard basis vector, the associated lateral matrix is indeed $L_j$). By analogy with adjacency matrices of graphs, $L_v$ is a natural analogue of the neighborhood of a vertex in a graph. Correspondingly, we get a notion of ``degree,'' which we may define as
\begin{eqnarray*}
\deg_\tA(v) & := & \rk L_v = \rk (\sum_{j=1}^n v_j L_j) 
=  \dim \linspan\{L_v \vec{w} : \vec{w} \in \F^m \} 
= \dim \linspan\{\vec{u}^t L_v : \vec{u} \in \F^\ell \}
\end{eqnarray*}
The last two characterizations are analogous to the fact that the degree of a vertex $v$ in a graph $G$ may be defined as the number of ``in-neighbors'' (nonzero entries the corresponding row of the adjacency matrix) or the number of ``out-neighbors'' (nonzero entries in the corresponding column). 

To ``individualize'' $v$, 
we can enlarge $\tA$ with a gadget to increase $\deg_\tA(v)$, as in the graph case. Note that $\deg_\tA(v) \leq \min\{\ell,m\}$ because the lateral matrices are all of size $\ell \times m$. For notational simplicity, let us individualize $v = \vec{e_1} = (1,0,\dotsc,0)^t$. To individualize $v$, we will increase its degree by $d = \min\{\ell,m\}+1 > \max_{v \in \F^n} \deg_\tA (v)$. Extend $\tA$ to a new 3-way array $\tA_v$ of size $(\ell+d) \times n \times (m+d)$; in the ``first'' $\ell \times n \times m$ ``corner'', we will have the original array $\tA$, and then we will append to it an identity matrix in one slice to increase $\deg(v)$. More specifically, the lateral slices of $\tA_v$ will be 
\[
L_1' = \begin{bmatrix} L_1 & 0 \\ 0 & I_d \end{bmatrix} \qquad \text{ and } \qquad L_j' = \begin{bmatrix} L_j & 0 \\ 0 & 0 \end{bmatrix} \quad ( \text{for } j > 1).
\]
Now we have that $\deg_{\tA_v}(v) \geq d$. This almost does what we want, but now note that any vector $w=(w_1,\dotsc,w_n)$ with $w_1 \neq 0$ has $\deg_{\tA_v}(w) =\rk (w_1 L_1' + \sum_{j \geq 2} w_j L_j) \geq d$. We can nonetheless consider this a sort of linear-algebraic individualization.

Leveraging this trick, we can then individualize an entire basis of $\F^n$ 
simultaneously, so that $d \leq \deg(v) < 2d$ for any vector $v$ in our basis, and 
$\deg(v') \geq 2d$ for any nonzero $v'$ outside the basis   
(not a scalar multiple of one of the basis vectors), as we do in the following 
proof of \Prop{prop:MonCodeEq}. 
This is 
also a 3-dimensional analogue of the reduction from \GI to \CodeEq \cite{Luks,miyazakiCodeEq,PR97} 
(where they use Hamming weight instead of rank).

\begin{proof}[Proof of \Prop{prop:MonCodeEq}]
Without loss of generality we assume $d>1$, as the problem is easily solvable when $d=1$. 
We treat a $d \times n$ matrix $A$ as a 3-way array of size $d \times n \times 1$, 
and then follow the outline proposed above, of individualizing the entire standard 
basis $\vec{e_1}, \dotsc, \vec{e_n}$. Since the third direction only has length 1, 
the maximum degree of any column is 1, so it suffices to use gadgets of rank 2. 
More specifically, 
we build a $(d + 2n) \times n \times (1 + 2n)$ 3-way array $\tA$ whose lateral slices are
\[
L_j = \left[\begin{array}{ccccccc}
a_{1,j} & \vzero_{1 \times 2} & \vzero_{1 \times 2} & \cdots & \vzero_{1 \times 2} & \cdots & \vzero_{1 \times 2} \\
\vdots & \vdots & \vdots & \ddots & \vdots & \ddots & \vdots\\
a_{d,j} & \vzero_{1 \times 2} & \vzero_{1 \times 2} & \cdots & \vzero_{1 \times 2}  & \cdots & \vzero_{1 \times 2} \\
\vzero_{2 \times 1} & \vzero_{2 \times 2} & \vzero_{2 \times 2} & \cdots & \vzero_{2 \times 2}  & \cdots & \vzero_{2 \times 2} \\
\vdots & \vdots & \vdots & \ddots & \vdots & \ddots & \vdots \\
\vzero_{2 \times 1} & \vzero_{2 \times 2} & \vzero_{2 \times 2} & \cdots & I_2 & \cdots & \vzero_{2 \times 2} \\
\vdots & \vdots & \vdots & \ddots & \vdots & \ddots & \vdots \\
\vzero_{2 \times 1} & \vzero_{2 \times 2} & \vzero_{2 \times 2} & \cdots & \vzero_{2 \times 2}  & \cdots & \vzero_{2 \times 2} \\
\end{array}\right]
\]
where the $I_2$ block is in the $j$-th block of size 2 (that is, rows $d + 2(j-1) + \{1,2\}$ and columns $2(j-1) + \{1,2\}$).
It will also be useful to visualize the frontal slices of $\tA$, as follows. Here each entry of the ``matrix'' below is actually a $(1+2n)$-dimensional vector, ``coming out of the page'':
$$
\tA=\begin{bmatrix}
\tilde a_{1,1} & \tilde a_{1,2} & \dots & \tilde a_{1,n} \\
\vdots & \vdots & \ddots & \vdots \\
\tilde a_{d,1} & \tilde a_{d,2} & \dots & \tilde a_{d,n}\\
e_{1,1} & \vzero & \dots & \vzero \\
e_{1,2} & \vzero & \dots & \vzero \\
\vzero & e_{2,1} & \dots & \vzero \\
\vzero & e_{2,2} & \dots & \vzero \\
\vdots & \vdots & \ddots & \vdots \\
\vzero & \vzero & \dots & e_{n,1} \\
\vzero & \vzero & \dots & e_{n,2}
\end{bmatrix}, 
\begin{array}{rcl}
\multicolumn{3}{c}{\text{where}} \\
\tilde a_{i,j} & = & \begin{bmatrix} a_{i,j} \\ \vzero_{2n \times 1} \end{bmatrix} \in \F^{1 + 2n} \\
e_{i,j} & =& \vec{e}_{1+2(i-1)+j} \in \F^{1+2n} \text{ for } i\in[n], j\in[2] \\ \\
\multicolumn{3}{c}{\text{ and the frontal slices are}} \\ \\
A_1 & = & \begin{bmatrix} A \\ \vzero_{2n \times n} \end{bmatrix} \\
A_{1 + 2(i-1) + j} & = & E_{d + 2(i-1) + j, i} \qquad \text{ for } i\in[n], j\in[2]
\end{array}
$$
(In $\tA$ we turn the vectors $\tilde a_{i,j}$ and $e_{i,j}$ ``on their side'' so they become perpendicular to the page. )

We claim that $A$ and $B$ are monomially equivalent as codes if and only if $\tA$ 
and $\tB$ are isomorphic as 3-tensors.

($\Rightarrow$) Suppose $QADP = B$ where $Q \in \GL(n,\F)$, $D = \diag(\alpha_1, \dotsc, \alpha_n)$ and $P \in S_n \leq \GL(n,\F)$. Then by examining the frontal slices it is not hard to see that for $Q' = \begin{bmatrix} Q & 0 \\ 0 & (DP)^{-1} \otimes I_2 \end{bmatrix}$ (where $DP^{-1} \otimes I_2$ denotes a $2n \times 2n$ block matrix, where the pattern of the nonzero blocks and the scalars are governed by $(DP)^{-1}$, and each $2 \times 2$ block is either zero or a scalar multiple of $I_2$) we have $Q' A_1 DP = B_1$ and $Q' A_{1 + 2(i-1) + j} DP = B_{1 + 2(\pi(i)-1) + j}$, where $\pi$ is the permutation corresponding to $P$. Thus $\tA$ and $\tB$ are isomorphic tensors, via the isomorphism $(Q', DP, \diag(I_1, P))$.

($\Leftarrow$) 
Suppose there exist $Q\in \GL(d+2n, \F)$, $P\in \GL(n, \F)$, 
and $R\in \GL(1+2n, \F)$, such that $Q\tA P =\tB^R$. 
First, note that every
lateral slice of $\tA$ is of rank either $2$ or $3$, 
and the actions of $Q$ and $R$ do 
not change the ranks of the lateral slices.
Furthermore, any non-trivial linear 
combination of more than $1$ lateral slice results in a lateral matrix of rank 
$\geq 4$. It follows that $P$ cannot take nontrivial linear combinations of the lateral slices, hence it must be monomial. 

Now consider the frontal 
slices. Note that, as we assume $d>1$, every frontal slice of $Q\tA P$, except the 
first one, is of 
rank $1$. Therefore, $R$ must be of the form 
$\begin{bmatrix} r_{1,1} & \vzero_{1 \times (n-1)} \\ \vec{r'} & R' \end{bmatrix}$ 
where $R'$ is $(n-1) \times (n-1)$. Since $R$ is invertible, we must have $r_{1,1}\neq 0$, and the first frontal slice of $\tB^R$ 
contains all the rows of $B$ scaled by $r_{1,1}$ in its first $d$ rows. The first frontal slice of $Q\tA P$ is a matrix that generates, by definition (and since we've shown $P$ is monomial), a code monomially equivalent to $A$. Since the first frontal slices of $Q \tA P$ and $\tB^R$ are equal, and the latter is just a scalar multiple of $B_1$, we have that $A$ and $B$ are monomially equivalent as codes as well.
\end{proof}

